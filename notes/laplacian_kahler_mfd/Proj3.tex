%!TEX program = Xelatex
\documentclass{ctexart}

\usepackage{amsmath, amssymb, amsthm, titlesec}
\usepackage[skins, breakable, theorems]{tcolorbox}
\usepackage{xparse, ocgx2, hyperref}
\usepackage[top=2.54cm, bottom=2.54cm, left=3.18cm, right=3.18cm]{geometry}

\def\Id{\,\mathrm{d}} % 积分中的正体d
\newcommand{\norm}[1]{\left|#1\right|} % 范数

% [number within=section/...]{}{<display name>}{<style>}{<lebal>(cite as "Thm:...")}
\newtcbtheorem[]{tcbdefinition}{Definition}{fonttitle = \bfseries}{Def}
\newtcbtheorem[]{tcbtheorem}{Theorem}{fonttitle = \bfseries}{Thm}
\newtcbtheorem[]{tcbproposition}{Proposition}{fonttitle = \bfseries}{Pro}
\newtcbtheorem[]{tcblemma}{Lemma}{fonttitle = \bfseries}{Lem}
\newtcbtheorem[]{tcbcorollary}{Corollary}{fonttitle = \bfseries}{Cor}
\NewDocumentEnvironment{definition}{ O{} O{} } % two optional arguments
  {\tcbdefinition{#1}{#2}}
  {\endtcbdefinition}
\NewDocumentEnvironment{theorem}{ O{} O{} }
  {\tcbtheorem{#1}{#2}}
  {\endtcbtheorem}
\NewDocumentEnvironment{proposition}{ O{} O{} }
  {\tcbproposition{#1}{#2}}
  {\endtcbproposition}
\NewDocumentEnvironment{lemma}{ O{} O{} }
  {\tcblemma{#1}{#2}}
  {\endtcblemma}
\NewDocumentEnvironment{corollary}{ O{} O{} }
  {\tcbcorollary{#1}{#2}}
  {\endtcbcorollary}
\makeatletter
\newcommand\tcb@cnt@tcbdefinitionautorefname{Definition}
\newcommand\tcb@cnt@tcbtheoremautorefname{Theorem}
\newcommand\tcb@cnt@tcbpropositionautorefname{Proposition}
\newcommand\tcb@cnt@tcblemmaautorefname{Lemma}
\newcommand\tcb@cnt@tcbcorollaryautorefname{Collory} 
\makeatother

% Sketch of proof & Remark
\newtcolorbox{sop}{
    blanker, breakable, left = 5mm,
    before skip = 10pt, after skip = 10pt,
    borderline west = {1mm}{0pt}{red},
    coltitle = black, title = \emph{Sketch of Proof}
}
\newtcolorbox{remark}{
    colback = white, fonttitle = \bfseries,
    colbacktitle = white, coltitle = black, enhanced,
    boxed title style = {sharp corners},
    attach boxed title to top left={yshift = -2mm, xshift = 5mm},
    title = Remark
}

\newcommand{\linktoprf}[1]{\hyperlink{#1}{To complete proof.}}
\newcommand{\targetprf}[1]{\hypertarget{#1}{Proof of \autoref{#1}}}

\title{Note 3: The Laplacian operator on Kahler manifolds}
\date{} % 显示日期

\renewcommand\refname{Reference}

%%%%%%%%%%%%%%%%%%%%%%%%%%%%%%%%%%%%%%%%%%%%%%%%%%%%%
\begin{document}

\maketitle

It is known that isothermal coordinates exist on an arbitrary surface with a Riemannian metric. Suppose the metric is given by 
$F(u, v)(d u \otimes d u+d v \otimes d v)$ and let $x=x(u, v), y=y(u, v)$ be a coordinate transformation. Denoting 
$(d u)^2=d u \otimes d u$, $d u d v=\frac{1}{2}(d u \otimes d v+d v \otimes d u)$ and $(d v)^2=d v \otimes d v$, then we have
$$
\begin{aligned}
&F(u, v)(d u  d u+d v  d v) \\
&=F(x, y)\big((x_u^2+y_u^2) d u^2+2(x_u x_v+y_u y_v) d u  d v +(x_v^2+y_v^2) d v^2\big).
\end{aligned}
$$
This leads us to 
$$
\left\{\begin{array}{c}y_u=-x_v, \\ x_u=y_v. \end{array}\right.
$$ 
Notice that $x=x(u, v), y=y(u, v)$ are harmonic maps. Therefore we can get a holomorphic chart transformation by letting $z=x+i y$. 
So that we obtain a complex structure on the surface and meanwhile, we have $F(u, v)(d u  d u+d v  d v) = F(u,v)dzd\bar{z}$.

From the above discussion, we can choose a local coordinate chart $(U, z)(z(p)=0)$, such that metric $g$ can be written as: 
$$
g=e^{2 \varphi(z, \bar{z})}|d z|^2, \quad \forall z \in U.
$$
Here $z=x+i y$ and the differential operators are given by  
\begin{equation*}
  \frac{\partial}{\partial z}=\frac{1}{2}\left(\frac{\partial}{\partial x}-i \frac{\partial}{\partial y}\right), 
  \quad \frac{\partial}{\partial \bar{z}}=\frac{1}{2}\left(\frac{\partial}{\partial x}+i \frac{\partial}{\partial y}\right).
\end{equation*}

First we calculate $K_{g, z z}$ in equation (2). To do so, we will first introduce complex connection, curvature and Laplacian operator 
on K\"ahler manifolds (complex manifolds with closed fundamental form), which can be found in \cite{Szekelyhidi2014}. 

Given a Kähler manifold $(M, g)$, we extend the Levi-Civita connection $\nabla$ by complex linearity and use it to differentiate tensor fields. 
By definition this satisfies $\nabla g=0$. In holomorphic normal coordinates the complex structure $J$ is constant, so we obtain $\nabla J=$ 0 , 
and since the fundamental form $\omega(X, Y)=g(J X, Y)$, we also have $\nabla \omega=0$. In terms of local holomorphic coordinates $z^1, \ldots, z^n$, 
we will use the following notation for the different derivatives:
$$
\nabla_i=\nabla_{\partial / \partial z^i}, \quad \nabla_{\bar{i}}=\nabla_{\partial / \partial \bar{z}^i}, \quad \partial_i=\frac{\partial}{\partial z^i}, 
\quad \partial_{\bar{i}}=\frac{\partial}{\partial \bar{z}^i} .
$$
Since
$$
J\left(\nabla_j \frac{\partial}{\partial z^k}\right)=\nabla_j J\left(\frac{\partial}{\partial z^k}\right)=\sqrt{-1} \nabla_j \frac{\partial}{\partial z^k},
$$
the vector field $\nabla_j \frac{\partial}{\partial z^k}$ has type $(1,0)$, and so we can define the Christoffel symbols $\Gamma_{j k}^i$ by
$$
\nabla_j \frac{\partial}{\partial z^k}=\sum_i \Gamma_{j k}^i \frac{\partial}{\partial z^i} .
$$
For the same reason $\nabla_{\bar{i}} \frac{\partial}{\partial z^k}$ also has type $(1,0)$, while $\nabla_k \frac{\partial}{\partial \bar{z}^i}$ has type $(0,1)$. 
However, since the connection is torsion free,
$$
\nabla_{\bar{i}} \frac{\partial}{\partial z^k}=\nabla_k \frac{\partial}{\partial \bar{z}^i},
$$
so both vector fields have to vanish. In addition, by $\bar{\partial}f=\overline{\partial\bar{f}}$, $\nabla_{\bar{i}} T=\overline{\nabla_i \bar{T}}$ for any tensor $T$, 
so the connection is determined completely by the coefficients $\Gamma_{j k}^i$. Note that
$$
\Gamma_{j k}^i=\Gamma_{k j}^i \quad \text { and } \quad \Gamma_{\bar{j} \bar{k}}^{\bar{i}}=\overline{\Gamma_{j k}^i} .
$$

Therefore, for a smooth function $f$, we have
$$
\begin{aligned}
\nabla^2 f &=\nabla\left(\frac{\partial f}{\partial z^i} d z^i + \frac{\partial f}{\partial \bar{z}^i} d \bar{z}^i\right) \\
&=\frac{\partial^2 f}{\partial z^i \partial z^j} d z^i d z^j + 2\frac{\partial^2 f}{\partial z^i \partial \bar{z}^j} d z^i d \bar{z}^j 
  + \frac{\partial^2 f}{\partial \bar{z}^i \partial \bar{z}^j} d \bar{z}^i d \bar{z}^j \\
&\quad  +\frac{\partial f}{\partial z^i} \nabla\left(d z^i\right)+\frac{\partial f}{\partial \bar{z}^i} \nabla\left(d \bar{z}^i\right) \\
% &=\frac{\partial^2 f}{\partial z^i \partial z^j} d z^i \otimes d z^j-\frac{\partial f}{\partial z^i} \Gamma_{j k}^i d z^j \otimes d z^k \\
&=\left(\frac{\partial^2 f}{\partial z^i \partial z^j}-\frac{\partial f}{\partial z^k} \Gamma_{i j}^k\right) d z^i d z^j 
  + 2\frac{\partial^2 f}{\partial z^i \partial \bar{z}^j} d z^i d \bar{z}^j \\
&\quad  + \left(\frac{\partial^2 f}{\partial \bar{z}^i \partial \bar{z}^j}-\frac{\partial f}{\partial \bar{z}^k} \overline{\Gamma_{i j}^k}\right) 
  d \bar{z}^i d \bar{z}^j.
\end{aligned}
$$

\begin{proposition}
  In terms of the metric $g_{j \bar{k}}$ the Christoffel symbols are given by
  $$
  \Gamma_{j k}^i=g^{i \bar{l}} \partial_j g_{k \bar{l}},
  $$
  where $(g^{i \bar{l}})$ is the matrix inverse to $(g_{i \bar{l}})$.
\end{proposition}
\begin{proof}[Proof]
  The Levi-Civita connection satisfies $\nabla g=0$. In coordinates this means
  $$
  0=\nabla_j g_{k \bar{l}}=\partial_j g_{k \bar{l}}-\Gamma_{j k}^p g_{p \bar{l}},
  $$
  thus
  $$
  g^{i \bar{l}} \partial_j g_{k \bar{l}}=\Gamma_{j k}^p g_{p \bar{l}} g^{i \bar{l}}=\Gamma_{j k}^p \delta_p^i=\Gamma_{j k}^i.
  $$
\end{proof}

We are now able to calculate  $K_{g, z z}$. Since the metric is given by $g=e^{2 \varphi(z, \bar{z})}d z d \bar{z}$, one can obtain 
$\Gamma_{1 1}^1= 2\frac{\partial \varphi}{\partial z}$ and $\Gamma_{1 1}^2=0$. So we have
$$
K_{g, z z}=\frac{\partial^2 K_g}{\partial z^2}-2 \frac{\partial K_g}{\partial z} \frac{\partial \varphi}{\partial z}.
$$

Covariant derivatives do not commute in general, and the failure to commute is measured by the curvature. The curvature is defined by
$$
\left(\nabla_k \nabla_{\bar{l}}-\nabla_{\bar{l}} \nabla_k\right) \frac{\partial}{\partial z^i}=R_{i k \bar{l}}^j \frac{\partial}{\partial z^j}, 
$$
while $\nabla_k$ commutes with $\nabla_l$ and $\nabla_{\bar{k}}$ commutes with $\nabla_{\bar{l}}$. 

In terms of the Christoffel symbols we can compute
$$
R_{i k \bar{l}}^j=-\partial_l \Gamma_{k i}^j,
$$
from which we find that in terms of the metric
$$
R_{i \bar{j} k \bar{l}}=-\partial_k \partial_{\bar{l}} g_{i \bar{j}}+g^{p \bar{q}}\left(\partial_k g_{i \bar{q}}\right)\left(\partial_{\bar{l}} g_{p \bar{j}}\right). 
$$

The Ricci curvature is defined to be the contraction
$$
R_{i \bar{j}}=g^{k \bar{l}} R_{i \bar{j} k \bar{l}},
$$
and the scalar curvature is
$$
R=g^{i \bar{j}} R_{i \bar{j}}.
$$
\begin{lemma}
  In local coordinates
  $$
  R_{i \bar{j}}=-\partial_i \partial_{\bar{j}} \ln \operatorname{det}\left(g_{p \bar{q}}\right) .
  $$
\end{lemma}
\begin{proof}[Proof]
  Using the formulas above, we have
  $$
  \begin{aligned}
  -\partial_{\bar{j}} \partial_i \ln \operatorname{det}\left(g_{p \bar{q}}\right) &=-\partial_{\bar{j}}\left(g^{p \bar{q}} \partial_i g_{p \bar{q}}\right) \\
  &=-\partial_{\bar{j}} \Gamma_{i p}^p \\
  &=R_{p i \bar{j}}^p \\
  &=R_{i \bar{j}}
  \end{aligned}
  $$
\end{proof}

From Lemma 1, one can calculate the scalar curvature, which is given by $-2\frac{\partial\bar{\partial}\varphi}{e^{2\varphi}}$. 
It is one-half of the Gauss curvature, which coincides with “using one half of the usual Riemannian Laplacian” in the sequel.  

Now we calculate the Gauss curvature. Let $\omega^1=e^{\varphi} d x, \omega^2=e^{\varphi} d y$, then $g=\omega^1 \omega^1+\omega^2 \omega^2$. 
By the structure equations, we have 
$$
\omega^1_2=\frac{(e^{\varphi})_y}{e^{\varphi}} d x-\frac{(e^{\varphi})_x}{e^{\varphi}} d y= \varphi_y d x-\varphi_x d y.
$$
So that Gauss curvature is given by 
$$
\begin{aligned}
K_g &=\frac{d \omega^1_2}{\omega^1 \wedge \omega^2} \\
&=-\frac{1}{e^{2\varphi} d x \wedge d y}\left(\varphi_{x x}+\varphi_{y y}\right) d x \wedge d y \\
&=-\frac{1}{e^{2\varphi}}\left(\varphi_{x x}+\varphi_{y y}\right).
\end{aligned}
$$
We denote the curvature by
$$
K_g=-\frac{\triangle \varphi}{e^{2 \varphi}},
$$
where "$\triangle$" is the real Laplacian operator with respect to the local Euclidean metric. 

Last before we give the proof, we define the Laplacian differential operator on K\"ahler manifolds, where we will use one-half of 
the usual Riemannian Laplacian (to understand "one-half" here, see \cite{Szekelyhidi2014}, p.33, and \cite{Huybrechts2005}, pp.114-116), which can be written 
in terms of local holomorphic coordinates as
$$
\triangle f=g^{k \bar{l}} \nabla_k \nabla_{\bar{l}} f=g^{k \bar{l}} \partial_k \partial_{\bar{l}} f. 
$$
Recall that $\nabla_k(\partial/\partial\bar{z}^l) = 0$, so the expression using partial derivatives holds even if we are not using normal coordinates, 
in contrast to the Riemannian case.


\bibliographystyle{unsrt}
\bibliography{NoteRef}

\end{document}