%!TEX program = Xelatex
\documentclass{ctexart}

\usepackage{amsmath, amssymb, amsthm, titlesec}
\usepackage[skins, breakable, theorems]{tcolorbox}
\usepackage{xparse, ocgx2, hyperref, endnotes}
\usepackage[top=2.54cm, bottom=2.54cm, left=3.18cm, right=3.18cm]{geometry}

\def\Id{\,\mathrm{d}} % 积分中的正体d
\newcommand{\norm}[1]{\left|#1\right|} % 范数

% [number within=section/...]{}{<display name>}{<style>}{<label>(cite as "Thm:...")}
\newtcbtheorem[]{tcbdefinition}{Definition}{fonttitle = \bfseries}{Def}
\newtcbtheorem[]{tcbtheorem}{Theorem}{fonttitle = \bfseries}{Thm}
\newtcbtheorem[]{tcbproposition}{Proposition}{fonttitle = \bfseries}{Pro}
\newtcbtheorem[]{tcblemma}{Lemma}{fonttitle = \bfseries}{Lem}
\newtcbtheorem[]{tcbcorollary}{Corollary}{fonttitle = \bfseries}{Cor}
\NewDocumentEnvironment{definition}{ O{} O{} } % two optional arguments
  {\tcbdefinition{#1}{#2}}
  {\endtcbdefinition}
\NewDocumentEnvironment{theorem}{ O{} O{} }
  {\tcbtheorem{#1}{#2}}
  {\endtcbtheorem}
\NewDocumentEnvironment{proposition}{ O{} O{} }
  {\tcbproposition{#1}{#2}}
  {\endtcbproposition}
\NewDocumentEnvironment{lemma}{ O{} O{} }
  {\tcblemma{#1}{#2}}
  {\endtcblemma}
\NewDocumentEnvironment{corollary}{ O{} O{} }
  {\tcbcorollary{#1}{#2}}
  {\endtcbcorollary}
\makeatletter
\newcommand\tcb@cnt@tcbdefinitionautorefname{Definition}
\newcommand\tcb@cnt@tcbtheoremautorefname{Theorem}
\newcommand\tcb@cnt@tcbpropositionautorefname{Proposition}
\newcommand\tcb@cnt@tcblemmaautorefname{Lemma}
\newcommand\tcb@cnt@tcbcorollaryautorefname{Corollary} 
\makeatother

% Sketch of proof & Remark
\newtcolorbox{sop}{
    blanker, breakable, left = 5mm,
    before skip = 10pt, after skip = 10pt,
    borderline west = {1mm}{0pt}{red},
    coltitle = black, title = \emph{Sketch of Proof}
}
\newtcolorbox{remark}{
    colback = white, fonttitle = \bfseries,
    colbacktitle = white, coltitle = black, enhanced,
    boxed title style = {sharp corners},
    attach boxed title to top left={yshift = -2mm, xshift = 5mm},
    title = Remark
}

\newcommand{\linktoprf}[1]{\hyperlink{#1}{To complete proof.}}
\newcommand{\targetprf}[1]{\hypertarget{#1}{Proof of \autoref{#1}}}

\title{Note 7: Comparison Theorems in Riemannian Geometry}
\date{} % 显示日期

\renewcommand\refname{Reference}

%%%%%%%%%%%%%%%%%%%%%%%%%%%%%%%%%%%%%%%%%%%%%%%%%%%%%
\begin{document}

\maketitle

\subsection*{Jacobi field and conjugate points}

In the Euclidean plane, geodesics originating from the same point may move farther apart, or figuratively speaking, they "diverge". However, on a unit sphere, 
any two geodesics originating from the same point will intersect at the antipodal point, or in other words, they "converge".

To study this "divergence" or "convergence" on a Riemannian manifold, the usual approach is to embed the geodesics into a family of geodesics, 
by considering a "variation of geodesics" with a fixed starting point, and analyzing the corresponding variation vector field which is known as a Jacobi field. 

For any given geodesic variation $\alpha$, we now determine the differential equation satisfied by $V \left|_{\alpha(t, 0)}\right.$. 
By the induced covariant derivatives (see \cite{ChenWeiHuan2002}, pp.149-150, Example 8.2), we have
$$
\nabla_T V-\nabla_V T-\alpha_*\left(\left[\frac{\partial}{\partial t}, \frac{\partial}{\partial s}\right]\right)=0 .
$$
But $\left[\frac{\partial}{\partial t}, \frac{\partial}{\partial s}\right]=0$, so $\nabla_T V=\nabla_V T$. Therefore $\nabla_T \nabla_T V=\nabla_T \nabla_V T$. Since $\nabla_T T=0$, we may write
$$
\nabla_T \nabla_T V=\nabla_T \nabla_V T-\nabla_V \nabla_T T
$$
Using the definition of the curvature tensor and the fact that
$$
[T, V]=\nabla_T V-\nabla_V T=\alpha_*\left(\left[\frac{\partial}{\partial t}, \frac{\partial}{\partial s}\right]\right)=0
$$
we get the Jacobi equation $\nabla_T \nabla_T V=R(T, V) T.$
\begin{definition}[Jacobi fields]
  A vector field $J$, along a geodesic $\gamma$ with tangent vector $T$ satisfying the equation 
  $$
  \nabla_T \nabla_T J=R(T, J) T
  $$
  is called a Jacobi field. 
\end{definition}

Let $\left\{E_i(t)\right\}$ be orthogonal and parallel along $\alpha(t, 0)$. Then the Jacobi equation may be written as the linear second-order system of ordinary differential equations
$$
\left[\begin{array}{c}
\left\langle J, E_1\right\rangle \\
\vdots \\
\left\langle J, E_n\right\rangle
\end{array}\right]^{\prime \prime}=\left[\left\langle R\left(T, E_i\right) T, E_j\right\rangle \right]_{i j}
\left[\begin{array}{c}
\left\langle J, E_1\right\rangle \\
\vdots \\
\left\langle J, E_n\right\rangle
\end{array}\right].
$$
From the theory of ordinary differential equations it follows that the space of solutions of this system is $2 n$-dimensional and that there exists a unique solution with prescribed initial value and first derivative. This is equivalent to prescribing $J(0)$ and $J^{\prime}(0)=\left.\nabla_T J\right|_{t=0}$. 
Also, we know that the zeros of a Jacobi field must be discrete provided it is not constantly zero. Otherwise, there will be a limit point of zeros at which $J$ and $J^\prime$ are both $0$, which leads to $J\equiv 0$.

Notice that since $\nabla_T T=0$, we have
$$
\langle J, T\rangle^{\prime \prime}=\left\langle J^{\prime \prime}, T\right\rangle=\langle R(T, J) T, T\rangle=0 .
$$
Therefore any Jacobi field $J$ may be written uniquely as
$$
J=J^\perp+(a t+b) T
$$
where $\left\langle J^\perp, T\right\rangle \equiv 0$. This shows that those Jacobi fields, called normal Jacobi fields, perpendicular to $T$ are what we really care about. 
One can calculate the coefficients: 
$$
a= \langle J'(0), T(0)\rangle \quad \text{ and } \quad b= \langle J(0), T(0)\rangle. 
$$
Thus $J$ is a normal Jacobi field if and only if 
$$
J(0)\perp T(0) \quad \text{ and } \quad J'(0)\perp T(0). 
$$

Finally, if $J$ is a Jacobi field, then $J$ comes from a variation of geodesics.
% In fact, let $c(s)$ be a curve such that $c^{\prime}(0)=J(0)$, and let $T$ and $J^{\prime}(0)$ be extended to parallel fields along $c(s)$. 
% Then the variation field of $\exp _{c(s)}\left(t\left(T+s J^{\prime}(0)\right)\right)$ is a Jacobi field with the same initial conditions as $J$. Therefore it equals $J$ by the uniqueness theorem above.
\begin{proposition}
  Let $M$ be a Riemannian manifold, $\gamma:[0,1] \rightarrow M$ a geodesic, and $J$ a Jacobi field along $\gamma$. Then there exists 
  a geodesic variation $\alpha(t, s)$, where $\alpha(t, 0)=\gamma(t)$, such that 
  $$
  J(t)=V \left|_{\alpha(t, 0)}\right..
  $$
\end{proposition}
\begin{proof}[Proof]
  Choose a curve $\lambda(s), s \in(-\varepsilon, \varepsilon)$ in $M$ such that $\lambda(0)=$ $\gamma(0), \lambda^{\prime}(0)=J(0)$. 
  Using parallel transport, along $\lambda$ choose a vector field $W(s)$ with $W(0)=\gamma^{\prime}(0), \nabla_{\frac{\partial}{\partial s}}W(0)=J'(0)$. 
  Define $\alpha(s, t)=\exp _{\lambda(s)} t W(s)$ and verify that $V(0,0)=\lambda^{\prime}(0)=J(0)$ and
  $$
  \nabla_{\frac{\partial}{\partial t}} V(0,0)=\nabla_{\frac{\partial}{\partial s}} T(0,0)=\nabla_{\frac{\partial}{\partial s}}W(0)=J'(0).
  $$
\end{proof}

% \begin{proposition}
%   Let $\gamma:[0, a] \rightarrow M$ be a geodesic and let $J$ be a Jacobi field along $\gamma$ with $J(0)=0$. Put $\frac{D J}{d t}(0)=w$ and $\gamma^{\prime}(0)=v$. Consider $w$ as an element of $T_{a v}\left(T_{\gamma(0)} M\right)$ and construct a curve $v(s)$ in $T_{\gamma(0)} M$ with $v(0)=a v, v^{\prime}(0)=w$. 
%   Put $f(t, s)=\exp _p\left(\frac{t}{a} v(s)\right)$, $p=\gamma(0)$, and define a Jacobi field $\bar{J}$ by $\bar{J}(t)=\frac{\partial f}{\partial s}(t, 0)$. Then $\bar{J}=J$ on $[0, a]$.
% \end{proposition}
% \begin{proof}[Proof]
%   For $s=0$, we have
%   $$
%   \begin{aligned}
%   \frac{D}{d t} \frac{\partial f}{\partial s} & =\frac{D}{\partial t}\left(\left(d \exp _p\right)_{t v}(t w)\right)=\frac{D}{\partial t}\left(t\left(d \exp _p\right)_{t v}(w)\right) \\
%   & =\left(d \exp _p\right)_{t v}(w)+t \frac{D}{\partial t}\left(\left(d \exp _p\right)_{t v}(w)\right)
%   \end{aligned}
%   $$
%   Therefore, for $t=0$,
%   $$
%   \frac{D \bar{J}}{d t}(0)=\frac{D}{\partial t} \frac{\partial f}{\partial s}(0,0)=\left(d \exp _p\right)_0(w)=w
%   $$
%   Since $J(0)=\bar{J}(0)=0$ and $\frac{D J}{d t}(0)=\frac{D \bar{J}}{d t}(0)=w$, we conclude, from the uniqueness theorem, that $J=\bar{J}$.
% \end{proof}

\begin{corollary}
  Let $\gamma:[0, a] \rightarrow M$ be a geodesic. Then a Jacobi field $J$ along $\gamma$ with $J(0)=0$ is given by
  $$
  J(t)=\left(\exp_{\gamma(0)}\right)_{*\left(t \gamma^{\prime}(0)\right)} \left(t J^{\prime}(0)\right), \quad t \in[0, a] .
  $$
\end{corollary}

The Gauss lemma tells us that the exponential mapping preserves the orthogonality with radial geodesics, and the length along the radial geodesic is preserved. 
Meanwhile, the change in length of the tangent vector orthogonal to the radial geodesic under the exponential mapping can be attributed to the computation of the magnitude of the Jacobi field.
Now we are going to relate the rate of spreading of the geodesics that start from $p \in M$ with the curvature at $p$. 
\begin{proposition}
  Let $p \in M$ and $\gamma:[0, a] \rightarrow M$ be a geodesic with $\gamma(0)=p, \gamma^{\prime}(0)=v$. Let $w \in T_v\left(T_p M\right)$ and let $J$ be a Jacobi field along $\gamma$ given by
  $$
  J(t)=\exp_{t v}(t w), \quad 0 \leq t \leq a
  $$
  Then the Taylor expansion of $|J(t)|^2$ about $t=0$ is given by
  $$
  |J(t)|^2=|w|t^2+\frac{1}{3}\langle R(v, w) v, w\rangle t^4+o(t^4).
  $$
\end{proposition}
\begin{proof}[Proof]
  Since $J(0)=0, J^{\prime}(0)=w$, let 
  $$
  f(t)=\langle J(t), J(t)\rangle=|J(t)|^2
  $$
  and we have: 
  $$
  f^{\prime}(t)=2\left\langle J^{\prime}(t), J(t)\right\rangle, \quad f^{\prime \prime}(t)=2\left\langle J^{\prime \prime}(t), J(t)\right\rangle+2\left\langle J^{\prime}(t), J^{\prime}(t)\right\rangle .
  $$
  Then for the first three coefficients: 
  $$
  f(0)=0, \quad f^{\prime}(0)=0, \quad f^{\prime \prime}(0)=2|w|^2.
  $$
  Take covariant derivative of $f^{\prime \prime}(t)$ and we have 
  $$
  \begin{aligned}
  f^{\prime \prime \prime}(t) & =6\left\langle J^{\prime \prime}(t), J^{\prime}(t)\right\rangle+2\left\langle J^{\prime \prime \prime}(t), J(t)\right\rangle, \\
  f^{(4)}(t) & =8\left\langle J^{\prime \prime \prime}(t), J^{\prime}(t)\right\rangle+6\left\langle J^{\prime \prime}(t), J^{\prime \prime}(t)\right\rangle+2\left\langle J^{(4)}(t), J(t)\right\rangle .
  \end{aligned}
  $$
  By Jacobi equation $J^{\prime \prime}(t)= R\left(\gamma^{\prime}, J\right) \gamma^{\prime}$, 
  $$
  f^{\prime \prime \prime}(0)=\left.6\left\langle R\left(\gamma^{\prime}, J\right) \gamma^{\prime}, J^{\prime}\right\rangle\right|_{t=0}=0, \quad f^{(4)}(0)=\left.8\left\langle J^{\prime \prime \prime}(t), J^{\prime}(t)\right\rangle\right|_{t=0},
  $$
  and 
  $$
  \begin{aligned}
  &\left\langle J^{\prime \prime \prime}(t), J^{\prime}(t)\right\rangle=\left\langle D_{\gamma^{\prime}}\left( R\left(\gamma^{\prime}, J\right) \gamma^{\prime}\right), J^{\prime}\right\rangle \\
  &= \gamma^{\prime}\left(\left\langle R\left(\gamma^{\prime}, J\right) \gamma^{\prime}, J^{\prime}\right\rangle\right)-\left\langle R\left(\gamma^{\prime}, J\right) \gamma^{\prime}, J^{\prime \prime}\right\rangle \\
  &= \gamma^{\prime}\left(\left\langle R\left(\gamma^{\prime}, J^{\prime}\right) \gamma^{\prime}, J\right\rangle\right)-\left\langle R\left(\gamma^{\prime}, J\right) \gamma^{\prime}, J^{\prime \prime}\right\rangle \\
  &=\left\langle D_{\gamma^{\prime}}\left( R\left(\gamma^{\prime}, J^{\prime}\right) \gamma^{\prime}\right), J\right\rangle+\left\langle R\left(\gamma^{\prime}, J^{\prime}\right) \gamma^{\prime}, J^{\prime}\right\rangle
    -\left\langle R\left(\gamma^{\prime}, J\right) \gamma^{\prime}, J^{\prime \prime}\right\rangle .
  \end{aligned}
  $$
  Therefore, $f^{(4)}(0)=8\langle R(v, w) v, w\rangle$.
  From Taylor expansion, we have  
  $$
  \begin{aligned}
  f(t) & =f(0)+t f^{\prime}(0)+\frac{t^2}{2 !} f^{\prime \prime}(0)+\frac{t^3}{3 !} f^{\prime \prime \prime}(0)+\frac{t^4}{4 !} f^{(4)}(0)+o\left(t^4\right) \\
  & =|w|^2 t^2+\frac{1}{3}\langle R(v, w) v, w\rangle t^4+o\left(t^4\right). 
  \end{aligned}
  $$
\end{proof}

\begin{corollary}
  If $\gamma:[0, \ell] \rightarrow M$ is parametrized by arc length, (i.e. $|v|=1$ ). Letting $w \in T_v\left(T_p M\right)$ with $|w|=1$ satisfying $\langle w, v\rangle=0$, 
  and $J(t)=\exp_{t v}(t w)$, then we have 
  $$
  |J(t)|^2=t^2-\frac{1}{3} K(v, w) t^4+o(t^4), 
  $$
  where $\sigma$ is the plane generated by $v$ and $w$. Moreover, 
  \begin{equation}
    |J(t)|=t-\frac{1}{6} K(v, w) t^3+o(t^3). 
  \end{equation}
\end{corollary}
\begin{proof}[Proof]
  Since $v \perp w,|v|=|w|=1$, then $\langle R(v, w) v, w\rangle=-K(v, w)$. Therefore, 
  $$
  |J(t)|=t \sqrt{1-\frac{1}{3} K(v, w) t^2+o\left(t^2\right)} .
  $$
  Together with expansion
  $$
  \sqrt{1-x}=1-\frac{1}{2} x+o(x),
  $$
  one can prove the corollary. 
\end{proof}

We now recall the introduction at the beginning of this subsection. The expression (1) essentially contains the relation between geodesics and curvature. 
Indeed, considering the geodesic variation
$$
\alpha(t, s)=\exp t v(s), \quad t \in[0, \delta], \quad s \in(-\varepsilon, \varepsilon),
$$
where$v(s)$ is a curve in $T_p M$ with $|v(s)|=1, v(0)=v, v^{\prime}(0)=w$, we see that 
the rays $t \mapsto t v(s), t \in[0, \delta]$, that start from the origin 0 of $T_p M$, deviate from the ray $t \rightarrow t v(0)$ with the velocity
$$
\left|\left(\frac{\partial}{\partial s} t v(s)\right)(0)\right|=|t w|=t
$$
On the other hand, (1) tells us that the geodesics $t \mapsto \exp _p(t v(s))$ deviate from the geodesic $\gamma(t)=\exp _p t v(0)$ 
with a velocity that differs from $t$ by a term of the third order in $t$, given by $-\frac{1}{6} K(v, w) t^3$. This tells us that, locally, 
the geodesics spread apart less than the rays in $T_p M$, if $K(v, w)>0$, and that they spread apart more than the rays in $T_p M$, if $K(v, w)<0$. 
Actually, for $t$ small, the value $K(v, w) t^3$ furnishes an approximation for the extent of this spread with an error of order $t^3$.

Now we are going to turn to the relationship between the singularities of the exponential map and Jacobi fields. 
\begin{definition}[Conjugate points]
  Let $\gamma:[0, a] \rightarrow M$ be a geodesic. The point $\gamma\left(t_0\right)$ is said to be conjugate to $\gamma(0)$ along $\gamma, t_0 \in(0, a]$, 
  if there exists a Jacobi field $J$ along $\gamma$, not identically zero, with $J(0)=0=J\left(t_0\right)$. 
  % The maximum number of such linearly independent fields is called the multiplicity of the conjugate point $\gamma\left(t_0\right)$.
\end{definition}

Observe that if $\gamma\left(t_0\right)$ is conjugate to $\gamma(0)$, then $\gamma(0)$ is conjugate to $\gamma\left(t_0\right)$
% 3.2 REMARK. If the dimension of $M$ is $n$, there exist exactly $n$ linearly independent Jacobi fields along the geodesic $\gamma:[0, a] \rightarrow M$, which are zero at $\gamma(0)$. 
% This follows from the fact, easily checked, that the Jacobi fields $J_1, \ldots, J_k$ with $J_i(0)=0$ are linearly independent if and only if $J_1^{\prime}(0), \ldots, J_k^{\prime}(0)$ are linearly independent. 
% In addition, the Jacobi field $J(t)=t \gamma^{\prime}(t)$ never vanishes for $t \neq 0$ (see Rem. 2.2). From this we deduce that the multiplicity of a conjugate point never exceeds $n-1$.

The following proposition relates conjugate points with the singularities of the exponential map. 
\begin{proposition}
  Let $\gamma:[0, a] \rightarrow M$ be a geodesic and put $\gamma(0)=$ $p$. The point $q=\gamma\left(t_0\right), t_0 \in(0, a]$, is conjugate to $p$ along $\gamma$ if 
  and only if $v_0=t_0 \gamma^{\prime}(0)$ is a critical point of $\exp _p$. 
  % In addition, the multiplicity of $q$ as a conjugate point of $p$ is equal to the dimension of the kernel of the linear map $\left(d \exp _p\right)_{v_0}$.
\end{proposition}
\begin{proof}[Proof]
  The point $q=\gamma\left(t_0\right)$ is a conjugate point of $p$ along $\gamma$ if and only if there exists a non-zero Jacobi field $J$ along $\gamma$ 
  with $J(0)=$ $J\left(t_0\right)=0$. Let $v=\gamma^{\prime}(0)$ and $w=J^{\prime}(0)$. From Corollary 1, 
  $$
  J(t)=\exp_{*t v}(t w), t \in[0, a].
  $$
  Observe that $J$ is non-zero if and only if $w \neq 0$. Therefore, $q=\gamma\left(t_0\right)$ is conjugate to $p$ if and only if
  $$
  0=J\left(t_0\right)=\exp_{*t_0 v}\left(t_0 w\right), \quad w \neq 0,
  $$
  that is, if and only if, $t_0 v$ is a critical point of $\exp _p$. The first assertion is therefore proved.

  % The multiplicity of $q$ is equal to the number of linearly independent Jacobi fields $J_1, \ldots, J_k$ which are zero at 0 and at $t_0$. As is easy to verify, the fields $J_1, \ldots, J_k$ are linearly independent if and only if $J_1^{\prime}(0), \ldots, J_k^{\prime}(0)$ are linearly independent in $T_p M$. From the
  % construction above, the multiplicity of $q$ is equal to the dimension of the kernel of $\left(d \exp _p\right)_{t_0 v}$.
\end{proof}

\subsection*{Rauch Comparison Theorem}

With enough preparations, we now state Rauch comparison theorem. Then we will give a brief proof as $n=2$ given that the motivation of the proof 
in this case is most clear. For general situations, we need introduce Morse index forms to complete the proof which will be discussed in the sequel subsection. 

Let $M$ and $\widetilde{M}$ be two $n$-dimensional Riemannian manifolds with $p\in M$ and $\widetilde{p}\in \widetilde{M}$. 
Let $\varphi:T_pM\rightarrow T_{\widetilde{p}}\widetilde{M}$ be a linear isometry. Let $x\in T_pM$ and $\widetilde{x}=\varphi(x)$. 
Consider the two geodesics $\gamma:[0,1]\rightarrow M$ and $\widetilde{\gamma}:[0,1]\rightarrow \widetilde{M}$ determined by 
$\gamma(t)=\exp_{p} t x$ and $\widetilde{\gamma}(t)=\exp_{\widetilde{p}} t \widetilde{x}$, respectively. Let $X\in T_x(T_pM)$, and 
let $\widetilde{X}=\varphi_*(X)\equiv \varphi(X)$, where $\widetilde{X}\in T_{\widetilde{x}}(T_{\widetilde{p}}M)$. 
\begin{theorem}[Rauch comparison theorem]
  Assume that 

  (1) $\tilde{\gamma}$ does not have conjugate points on $(0, a]$, 

  (2) for all $t$ and all $x \in T_{\gamma(t)}(M), \tilde{x} \in T_{\tilde{\gamma}(t)}(\tilde{M})$, we have
  $$
  \tilde{K}\left(\tilde{x}, \tilde{\gamma}^{\prime}(t)\right) \geq K\left(x, \gamma^{\prime}(t)\right). 
  $$
  Then 
  $$
  |(\exp_{\widetilde{p}})_{*(t\tilde{x})}\widetilde{X}|\leq |(\exp_{p})_{*(tx)} X|.
  $$
\end{theorem}

If any plane containing $\gamma'(t)$ has a non-positive sectional curvature, then $\exp_{p}: T_pM \rightarrow M$ is a distance-expanding map, i.e.
$$
\left|(\exp_{p})_{*(tx)} X\right| \geq |X|, \quad \forall X \in T_pM.
$$
Also, if any plane containing $\gamma'(t)$ has a non-negative sectional curvature and $\gamma$ has no conjugate point to $p$, 
then $\exp_{p}: T_pM \rightarrow M$ is a distance-reducing map, i.e.
$$
\left|(\exp_{p})_{*(tx)} X\right| \leq |X|, \quad \forall X \in T_pM.
$$

Since $(\exp_{p})_{*(tx)}$ is always isometric along the direction of $x$, without loss of generality, we can assume that $X \perp x, \widetilde{X} \perp \widetilde{x}$ 
in $T_pM, T_{\widetilde{p}}\widetilde{M}$, respectively, and describe $(\exp_{p})_{*(tx)} X$ via Jacobi fields. 
% Let $\gamma_u(t) = \exp_{p} t(\widetilde{x}+uX)$ define the variation $\{\gamma_u\}$ (one-parameter family of geodesics), and let $\widetilde{J}(t)$ be 
% the restriction of the transverse vector field of $\{\gamma_u\}$ on $\gamma$. Then $\widetilde{J}$ is the Jacobi field along $\gamma$, 
% with $\widetilde{J}(0) = 0, \dot{\widetilde{J}}(0) = X, \widetilde{J}(1) = (\exp_{p})_* X$. Similarly, using $\widetilde{\gamma}_u(t) = \exp_{p} t(x+u\widetilde{X})$, 
% we have $J(t)$, which is the Jacobi field along $\widetilde{\gamma}$, with $J(0) = 0, \dot{J}(0) = \widetilde{X}, J(1) = (\exp_{p})_* \widetilde{X}$. 
By these observations, we can restate the Rauch comparison theorem with slight generalization of $\widetilde{M}$ as follows:
\begin{theorem}[Rauch Comparison Theorem]\label{RCT}
  Let $\gamma:[0, a] \rightarrow M^n$ and $\tilde{\gamma}:[0, a] \rightarrow \tilde{M}^{n+k}$, $k \geq 0$, be geodesics with the same velocity 
  (i.e. $\left.\left|\gamma^{\prime}(t)\right|=\left|\tilde{\gamma}^{\prime}(t)\right|\right)$. 
  Let $J$ and $\tilde{J}$ be normal Jacobi fields along $\gamma$ and $\tilde{\gamma}$, respectively, with $J(0)=\tilde{J}(0)=0$ such that
  $$
  \left|J^{\prime}(0)\right|=\left|\tilde{J}^{\prime}(0)\right| .
  $$
  Assume that 

  (1) $\tilde{\gamma}$ does not have conjugate points on $(0, a]$, 

  (2) for all $t$ and all $x \in T_{\gamma(t)}(M), \tilde{x} \in T_{\tilde{\gamma}(t)}(\tilde{M})$, we have
  $$
  \tilde{K}\left(\tilde{x}, \tilde{\gamma}^{\prime}(t)\right) \geq K\left(x, \gamma^{\prime}(t)\right). 
  $$
  Then
  $$
  |\tilde{J}| \leq|J|
  $$
  % In addition, if for some $t_0 \in(0, a]$, we have $\left|\tilde{J}\left(t_0\right)\right|=\left|J\left(t_0\right)\right|$, then $\tilde{K}\left(\tilde{J}(t), \tilde{\gamma}^{\prime}(t)\right)=K\left(J(t), \gamma^{\prime}(t)\right)$, for all $t \in\left[0, t_0\right]$.
\end{theorem}

\begin{proof}[Proof of Theorem 2 as $n=2$]
  First, we know that $J, \widetilde{J}$ are normal Jacobi fields. If we let $W(t), \widetilde{W}(t)$ be the unit vector fields parallel
  along $\gamma, \widetilde{\gamma}$, respectively, such that $W \perp \dot{\gamma}, \widetilde{W} \perp \dot{\widetilde{\gamma}}$, then we can write
  $$
   J(t)=f(t) W(t), \quad\widetilde{J}(t)=\tilde{f}(t) \widetilde{W}(t).
  $$
  Then the theorem becomes:
  Assuming $\tilde{f}, f$ are functions on $[0, a]$ satisfying the following equations:
  $$
  \begin{aligned}
  & \left\{\begin{array}{l}
  f^{\prime \prime}+K f=0, \\
  f(0)=0, \quad f^{\prime}(0)=b>0 ;
  \end{array}\right.
  & \left\{\begin{array}{l}
    \tilde{f}^{\prime \prime}+\tilde{K} \tilde{f}=0, \\
    \tilde{f}(0)=0, \quad \tilde{f}^{\prime}(0)=b>0 ;
    \end{array}\right. 
  \end{aligned}
  $$
  where $\tilde{K}, K$ are functions on $(0, a]$ and
  $$
  \tilde{K}(t) \geq K(t), \quad \forall t \in[0, a],
  $$
  then for any $\tilde{f}>0$ on $[0, a]$, we have $f \geq \tilde{f}$. 
  
  To prove this, we only need to let 
  $$
  \Phi(t) = (ff' - \tilde{f}'f)(t).
  $$ 
  It is known that when $\tilde{f},f \geq 0$, we always have $\Phi'(t) \geq 0$. 
  Guaranteed by $f'(0)>0$, there exists $0<r\leq a$ such that $f>0$ on $(0,r)$. Therefore, since $\Phi(0)=0$, we have
  $$
  \Phi(t) \geq 0, \quad \forall t \in [0,r].
  $$
  Then rewrite $\Phi \geq 0$ as
  \begin{equation}
    \frac{f' }{f} \geq \frac{\tilde{f}'}{\tilde{f}}
  \end{equation}
  and integrate it yields
  $$
  \ln\frac{f(t)}{f(\epsilon)} \geq \ln\frac{\tilde{f}(t)}{\tilde{f}(\epsilon)}
  $$
  where $0 < \epsilon < t < r$. Therefore,
  \begin{equation}
    \frac{f(t)}{\tilde{f}(t)} \geq \lim_{\epsilon \rightarrow 0} \frac{f(\epsilon)}{\tilde{f}(\epsilon)} = \frac{b}{b} = 1.
  \end{equation}
  Thus we have $f(t) \geq \tilde{f}(t), \forall t \in (0,r)$, and furthermore, $f \geq \tilde{f}$ on $[0,r]$.

  To illustrate one can set $r=a$, we prove that $f>0$ on $(0,a]$. Otherwise, there exists $c\in (0,a)$ such that $f(c)=0$ and $f(t)>0$ on $(0,c)$. 
  Taking $r=c$, repeat the above steps and we have: as $t\rightarrow c$ in limit (3),
  $$
  0=f(c)\geq \tilde{f}(c)>0, 
  $$
  which leads to a contradiction.
\end{proof}

\hypertarget{GSpre}{For general situations}, let 
$$
f(t)=\langle J(t),J(t)\rangle,\quad \widetilde{f}(t)=\langle \widetilde{J}(t),\widetilde{J}(t)\rangle.
$$
Motivated by above proof, we want to prove: 
$$
\lim_{\epsilon \rightarrow 0} \frac{f(\epsilon)}{\tilde{f}(\epsilon)} = 1, \quad \quad \frac{f' }{f} \geq \frac{\tilde{f}'}{\tilde{f}} \text{ on } (0,r)\quad \text{ and } \quad 
\tilde{f},f > 0 \text{ on } (0,a],
$$
where $0<r\leq a$ such that $f>0$ on $(0,r)$.
The first can be obtained by L'Hospital's rule:
$$
\begin{aligned}
\lim _{t \rightarrow 0} \frac{f(t)}{\tilde{f}(t)} 
& =\lim _{t \rightarrow 0} \frac{\left\langle J^{\prime}(t), J(t)\right\rangle}{\left\langle\tilde{J}^{\prime}(t), \tilde{J}(t)\right\rangle} \\
& =\lim _{t \rightarrow 0} \frac{\left\langle J^{\prime}(t), \frac{J(t)-J(0)}{t}\right\rangle}{\left\langle\tilde{J}^{\prime}(t), \frac{\tilde{J}(t)-\tilde{J}(0)}{t}\right\rangle} \\
% & =\lim _{t \rightarrow 0} \frac{\left\langle J^{\prime \prime}(t), J(t)\right\rangle+\left\langle J^{\prime}(t), J^{\prime}(t)\right\rangle}{\left\langle\tilde{J}^{\prime \prime}(t), \tilde{J}(t)\right\rangle+\left\langle\tilde{J}^{\prime}(t), \tilde{J}^{\prime}(t)\right\rangle} \\
& =\frac{\left|J^{\prime}(0)\right|^2}{\left|\tilde{J}^{\prime}(0)\right|^2}=1.
\end{aligned}
$$
To prove $\tilde{f},f > 0$ on $(0,a]$, assume that $\tilde{J}$ is a non-zero Jacobi field; otherwise, the conclusion is trivial. 
Since there are no conjugate points to $\tilde{\gamma}(0)$ along $\tilde{\gamma}$, the function $|\tilde{J}(t)|$ has no zeros other than $t=0$, 
i.e. $\tilde{f} > 0$ on $(0,a]$. On the other hand, since $|J'(0)|>0$, there exists $0<r\leq a$ such that $f>0$ on $(0,r)$. 
Then one can prove $f>0$ on $(0,a]$ in the same way as the case $n=2$, provided the second inequality holds for $(0,r)$.  
However, to verify this inequality, one needs the minimal property of Jacobi fields which is known as the basic index lemma.

% \begin{theorem}
%   Let $\gamma:[0, a] \rightarrow \widetilde{M}, \widetilde{\gamma}:[0, a] \rightarrow M$ be both normal geodesics, where $\operatorname{dim} \widetilde{M} = \operatorname{dim} M$. 
%   Let $\widetilde{J}, J$ be the Jacobi fields along $\gamma, \widetilde{\gamma}$, respectively, such that 
%   $\widetilde{J}(0) = J(0) = 0, \dot{\widetilde{J}}(0) \perp \dot{\gamma}(0), \dot{J}(0) \perp \dot{\widetilde{\gamma}}(0), |\dot{\widetilde{J}}(0)| = |\dot{J}(0)|$. 
%   Assuming that (1) and (2) in the previous theorem hold (i.e., replacing $[0,1]$ with $[0, a]$), then
%   $$
%   |J(t)| \geq |\widetilde{J}(t)|, \quad \forall t \in [0, a]
%   $$
% \end{theorem}

% Proof. Observe that, from Proposition 3.6 of Chapter 5, the condition $\left\langle J^{\prime}(0), \gamma^{\prime}(0)\right\rangle=\left\langle\tilde{J}^{\prime}(0), \tilde{\gamma}^{\prime}(0)\right\rangle$ is equivalent (with $J(0)=\tilde{J}(0)=$ $0)$ to the condition $\left\langle J, \gamma^{\prime}\right\rangle=\left\langle\tilde{J}, \tilde{\gamma}^{\prime}\right\rangle$. In addition, since
% $$
% \left\langle J, \gamma^{\prime}\right\rangle \gamma^{\prime}=\left\langle J^{\prime}(0), \gamma^{\prime}(0)\right\rangle t \gamma^{\prime}+\left\langle J(0), \gamma^{\prime}(0)\right\rangle \gamma^{\prime}
% $$
% the tangential components of $J$ and $\tilde{J}$ have, by hypothesis, the same length. Therefore, we can suppose that
% $$
% \left\langle J, \gamma^{\prime}\right\rangle=0=\left\langle\tilde{J}, \tilde{\gamma}^{\prime}\right\rangle
% $$

\subsection*{Morse index form and proof of Rauch comparison theorem}

To see how the Morse index form is motivated, we shall introduce "path space" and one can see more details in \cite{Milnor1973}, Part \uppercase\expandafter{\romannumeral3}. 
And for simplicity, we replace "the energy of a path" in \cite{Milnor1973} by the length of a path, which is more convenient to understand.

The set of all piecewise smooth paths from $p$ to $q$ in $M$ will be denoted by $\Omega(M ; p, q)$, or briefly by $\Omega(M)$ or $\Omega$.
We will think of $\Omega$ as being something like an "infinite dimensional manifold", for example, a topological space in which each point has a neighborhood 
homeomorphic to an open set in a Banach space. To start the analogy we make the following definition.

By the tangent space of $\Omega$ at a path $w$ will be meant the vector space consisting of all piecewise smooth vector fields $W$ along $\omega$ for which vanishes 
at endpoints. The notation $T_\omega \Omega$ will be used for this vector space.

If $F$ is a real valued function on $\Omega$ it is natural to ask what
$$
F_{*\omega}: T_\omega \Omega \rightarrow T_{F(\omega)} \mathbb{R}
$$
the induced map on the tangent space, should mean. When $F$ is a function which is smooth in the usual sense, on a Riemannian manifold $M$. 
Given $X \in T_p M$, choose a smooth path $u \rightarrow \alpha(u)$ in $M$, which is defined for $-\varepsilon<u<\varepsilon$, so that
$$
\alpha(0)=p, \quad \frac{d \alpha}{d u}(0)=X. 
$$
Then $F_*(X)$ is equal to 
$$
\left.\frac{d(F(\alpha(u))}{d u}\right|_{u=0},
$$
multiplied by the basis vector $\left(\frac{d}{d t}\right)_{F(p)} \in T_{F(p)}\mathbb{R}$. 

A fixed endpoints variation of $\omega$ denoted by $\bar{\alpha}(u)=\alpha(u,t)$ for $u\in (-\varepsilon,\varepsilon)$ may be considered as 
a "smooth path" in $\Omega$ (recall that a variation is a smooth map). And more generally if $u$ in the variation is defined on a neighborhood of $0\in\mathbb{R}^n$, 
then it is called an $n$-parameter variation of $\omega$. By analogy, if $F$ is a real valued function on $\Omega$, we attempt to define $F_{*\omega}$
% $$
% F_{*\omega}: T_\omega\Omega \rightarrow T_{F(\omega)}\mathbb{R}
% $$
as follows. Given $W \in T_\omega \Omega$ choose a variation $\bar{\alpha}:(-\varepsilon, \varepsilon) \rightarrow \Omega$ with
$$
\bar{\alpha}(0)=\omega,\quad \frac{d \bar{\alpha}}{d u}(0)=W
$$
and set $F_{* \omega}(W)$ equal to $\left.\frac{d(F(\bar{\alpha}(u))}{d u}\right|_{u=0}$ multiplied by the tangent vector
$\left(\frac{d}{d t}\right)_{F(\omega)}$. Of course without hypothesis on $F$ there is no guarantee that
this derivative will exist, or will be independent of the choice of $\bar{\alpha}$.
We will not investigate what conditions $F$ must satisfy but we have indicated how $F_*$ might be defined only to motivate the following.
\begin{definition}
  A path $\omega$ is a critical path for a function $F: \Omega\rightarrow \mathbb{R}$ if and only if for any variation of $\omega$,
  $$
  \left.\frac{d(F(\bar{\alpha}(u))}{d u}\right|_{u=0} = 0.
  $$
\end{definition}

We denote the length of the piecewise smooth curve $\gamma:[a, b] \rightarrow M$ by $L(\gamma)$. By definition,
$$
L(\gamma)=\int_a^b|\gamma^{\prime}(t)| \mathrm{d} t
$$
It follows from the chain rule that $L(\gamma)$ does not depend on a particular choice of parameterization. Then the length $L: \Omega\rightarrow\mathbb{R}$ is a real valued 
function on $\Omega$. By first variation of arc length (\cite{ChenWeiHuan2002}, Corollary 3.6, p.190), we know that the path $\omega$ is a critical point for the function $L$ if and only if $\omega$ is a geodesic.

Continuing with the analogy developed in the previous, we now wish to define a bilinear functional, called the Hessian of $L$ at $\gamma$
$$
L_{* * \gamma}: T_\gamma \Omega\times T_\gamma \Omega\rightarrow \mathbb{R}
$$
when $\gamma$ is a critical point of the function $L$, i.e. a geodesic. 

If $f$ is a real valued function on a Riemannian manifold $M$ with critical point $p$, then the Hessian
$$
\nabla^2 f: T_p M \times T_p M \rightarrow \mathbb{R}
$$
is a $(0,2)$-tensor can be defined as $\nabla^2 f=\nabla(df)$. Equivalently, given $X_1, X_2 \in T M_p$ choose a smooth map
$\alpha: U\rightarrow M$ where $U$ is a neighborhood of $(0,0)$ in $\mathbb{R}^2$ such that 
$$
\alpha(0,0)=p,\quad \frac{\partial \alpha}{\partial u_1}(0,0)=X_1,\quad \frac{\partial \alpha}{\partial u_2}(0,0)=X_2 .
$$
Then
$$
\nabla^2 f\left(X_1, X_2\right)=\left.\frac{\partial^2 f\left(\alpha\left(u_1, u_2\right)\right)}{\partial u_1 \partial u_2}\right|_{(0,0)}.
$$
This suggests defining $L_{* *}$ as follows. Given vector fields $W_1, W_2 \in T_\gamma\Omega$ choose a 2-parameter variation
$\alpha: U \times[a,b] \rightarrow M$, such that
$$
\alpha(0,0, t)=\gamma(t), \quad \frac{\partial \alpha}{\partial u_1}(0,0, t)=W_1(t), \quad \frac{\partial \alpha}{\partial u_2}(0,0, t)=W_2(t) .
$$
Then the Hessian $L_{* *}\left(W_1, W_2\right)$ will be defined to be the second partial derivative
$$
\left.\frac{\partial^2 L\left(\bar{\alpha}\left(u_1, u_2\right)\right)}{\partial u_1 \partial u_2}\right|_{(0,0)}
$$
where $\bar{\alpha}\left(u_1, u_2\right) \in \Omega$ denotes the path $\bar{\alpha}\left(u_1, u_2\right)(t)=\alpha\left(u_1, u_2, t\right)$. 
By second variation formula theorem (\cite{Cheeger2008}, pp.16-18), $L_{* * \gamma}(W_1,W_2)$ is well defined symmetric and bilinear function and depends only on $W_1,W_2$ to $\gamma$. 
This Hessian is a Morse index form and more generally, we write the index form as: 
$$
I_\gamma(V, W)=\int_a^b\left\langle\nabla_T V, \nabla_T W\right\rangle+\langle R(W, T) V, T\rangle\mathrm{d}\,t,
$$
where $V$ and $W$ are smooth vectors fields along $\gamma$. Sometimes, we simply write $I(V,W)$ or $I_a^b(V,W)$.
Since $\left\langle\nabla_T V, \nabla_T W\right\rangle=\frac{\mathrm{d}}{\mathrm{d}t}\left\langle\nabla_T V, W\right\rangle-\left\langle\nabla_T \nabla_T V,  W\right\rangle$, 
$I_\gamma(V, W)$ can be written as 
$$
I(V, W)=\left\langle\nabla_T V, W\right\rangle-\int_a^b\langle \nabla_T \nabla_T V - R(T, V) T, W\rangle\mathrm{d}\,t. 
$$
For cases of piecewise smooth ones, see \cite{Cheeger2008}, pp.16-18 as well.

In a certain sense, Jacobi fields minimize the index form. 
\begin{lemma}[Basic index lemma]
  Let $\gamma$ be a geodesic in $M$ from $p$ to $q$ containing no points conjugate to $p$. Let $W$ be a piecewise smooth vector field on $\gamma$ 
  and $V$ the unique Jacobi field such that $V(p)=W(p)=$ 0 and $V(q)=W(q)$. Then 
  $$
  I(V, V) \leq I(W, W), 
  $$
  and equality holds only if $V=W$.
\end{lemma}

We are now in the position to prove \underline{\hyperref[RCT]{Rauch Comparison Theorem}}.
\begin{proof}[Proof of Theorem 2]
  The conclusion we want to prove is:
  $$
  \frac{|J(t)|^2}{|\tilde{J}(t)|^2} \geq 1, \quad \forall t \in (0, a].
  $$
  By \underline{\hyperlink{GSpre}{previous analysis}}, we have 
  $$
  \lim _{t \rightarrow 0} \frac{|J(t)|^2}{|\tilde{J}(t)|^2}=1, \quad \text{and} \quad |\tilde{J}|, |J| > 0 \text{ on } (0, a].
  $$
  Therefore, we only need to prove:
  \begin{equation}
    \frac{\left\langle J^{\prime}, J\right\rangle}{\langle J, J\rangle} \geq \frac{\left\langle\tilde{J}^{\prime}, \tilde{J}\right\rangle}{\langle\tilde{J}, \tilde{J}\rangle}, \quad t \in (0, r),
  \end{equation}
  where $r \in (0, a]$, ensuring that the Jacobi field $J$ has no zeros on $(0, r)$.
  
  For any fixed $t_1 \in (0, r)$, since $J(t_1) \neq 0$, we can define
  $$
  W_{t_1}(t) = \frac{J(t)}{\left|J(t_1)\right|} \quad \text{and} \quad \tilde{W}_{t_1}(t) = \frac{\tilde{J}(t)}{\left|\tilde{J}(t_1)\right|}.
  $$
  Then, equation (4) holds at $t = t_1$ if and only if
  \begin{equation}
    \left.\left\langle W_{t_1}^{\prime}(t), W_{t_1}(t)\right\rangle\right|_{t=t_1} \geq \left.\left\langle\tilde{W}_{t_1}^{\prime}(t), \tilde{W}_{t_1}(t)\right\rangle\right|_{t=t_1}.
  \end{equation}
  Because $W_{t_1}$ and $\tilde{W}_{t_1}$ are Jacobi fields on $\gamma$ and $\tilde{\gamma}$, respectively, and both of which are zero at $t = 0$, we can write them as:
  $$
  \begin{aligned}
  & \left.\left\langle W_{t_1}^{\prime}(t), W_{t_1}(t)\right\rangle\right|_{t=t_1}=I_0^{t_1}\left(W_{t_1}, W_{t_1}\right), \\
  & \left.\left\langle\tilde{W}_{t_1}^{\prime}(t), \tilde{W}_{t_1}(t)\right\rangle\right|_{t=t_1}=\tilde{I}_0^{t_1}\left(\tilde{W}_{t_1}, \tilde{W}_{t_1}\right),
  \end{aligned}
  $$
  where $I$ and $\tilde{I}$ are the index forms along the geodesics $\gamma$ and $\tilde{\gamma}$ in $M$ and $\tilde{M}$, respectively. 
  Thus, the inequality to be proven in equation (5) is equivalent to an inequality involving the index forms:
  $$
  I_0^{t_1}\left(W_{t_1}, W_{t_1}\right) \geq \tilde{I}_0^{t_1}\left(\tilde{W}_{t_1}, \tilde{W}_{t_1}\right).
  $$

  In the following, we transform the vector field $W_{t_1}$ defined along $\left.\gamma\right|_{\left[0, t_1\right]}$ in $M$ 
  into a vector field $U(t)$ defined along $\left.\tilde{\gamma}\right|_{\left[0, t_1\right]}$ in $\tilde{M}$. This transformation ensures that:
  $$
  |U(t)| = \left|W_{t_1}(t)\right| \quad \text{and} \quad \left|U^{\prime}(t)\right| = \left|W_{t_1}^{\prime}(t)\right|,
  $$
  Furthermore, $U$ and $\tilde{W}_{t_1}$ take the same values at both ends of $\left.\tilde{\gamma}\right|_{\left[0, t_1\right]}$. With this setup, 
  we can use the Basic Index Lemma to compare $\tilde{I}_0^{t_1}(U, U)$ and $\tilde{I}_0^{t_1}\left(\tilde{W}_{t_1}, \tilde{W}_{t_1}\right)$. 
  Additionally, based on the assumptions about the sectional curvatures of $M$ and $\tilde{M}$, we can compare $\tilde{I}_0^{t_1}(U, U)$ and $I_0^{t_1}\left(W_{t_1}, W_{t_1}\right)$.

  Let $\left\{e_i(t)\right\}$ and $\left\{\tilde{e}_i(t)\right\}$ be unit orthogonal frames paralleling along $\gamma$ and $\tilde{\gamma}$, respectively, such that:
  $$
  \begin{array}{ll}
  e_1(t)=\gamma^{\prime}(t), & e_2\left(t_1\right)=W_{t_1}\left(t_1\right), \\
  \tilde{e}_1(t)=\tilde{\gamma}^{\prime}(t), & \tilde{e}_2\left(t_1\right)=\tilde{W}_{t_1}\left(t_1\right).
  \end{array}
  $$
  Since $J \perp \gamma^{\prime}$, we can express $W_{t_1}(t)$ as:
  $$
  W_{t_1}(t)=\sum_{i=2}^m W^i(t) e_i(t),
  $$
  where $W^i(t)$ satisfies:
  $$
  \begin{aligned}
  & W^2(0)=\cdots=W^m(0)=0, \\
  & W^2\left(t_1\right)=1, \quad W^3\left(t_1\right)=\cdots=W^m\left(t_1\right)=0.
  \end{aligned}
  $$
  Defining the vector field $U(t)$ along the geodesic $\tilde{\gamma}$ as:
  $$
  U(t)=\sum_{i=2}^m W^i(t) \tilde{e}_i(t),
  $$
  we have:
  $$
  U(0)=\tilde{W}_{t_1}(0)=0, \quad U\left(t_1\right)=\tilde{e}_2\left(t_1\right)=\tilde{W}_{t_1}\left(t_1\right).
  $$
  Since there are no conjugate points to $\tilde{\gamma}(0)$ along $\left.\tilde{\gamma}\right|_{\left[0, t_1\right]}$, according to the Basic Index Lemma:
  \begin{equation}
    \tilde{I}_0^{t_1}(U, U) \geq \tilde{I}_0^{t_1}\left(\tilde{W}_{t_1}, \tilde{W}_{t_1}\right).
  \end{equation}

  On the other hand, according to the definitions, we have:
  $$
  \begin{aligned}
  I_0^{t_1}\left(W_{t_1}, W_{t_1}\right) & =\int_0^{t_1}\left\{\left|W_{t_1}^{\prime}\right|^2+\left\langle\mathcal{R}\left(\gamma^{\prime}, W_{t_1}\right) \gamma^{\prime}, W_{t_1}\right\rangle\right\} \mathrm{d} t \\
  & =\int_0^{t_1}\left\{\left|W_{t_1}^{\prime}\right|^2-\left|W_{t_1}\right|^2 K\left(\gamma^{\prime}, W_{t_1}\right)\right\} \mathrm{d} t, \\
  \tilde{I}_0^{t_1}(U, U) & =\int_0^{t_1}\left\{\left|U^{\prime}\right|^2+\left\langle\tilde{\mathcal{R}}\left(\tilde{\gamma}^{\prime}, U\right) \tilde{\gamma}^{\prime}, U\right\rangle\right\} \mathrm{d} t \\
  & =\int_0^{t_1}\left\{\left|U^{\prime}\right|^2-|U|^2 \tilde{K}\left(\tilde{\gamma}^{\prime}, U\right)\right\} \mathrm{d} t .
  \end{aligned}
  $$
  Based on the construction of the vector field $U$, we know that:
  $$
  |U(t)|^2=\left|W_{t_1}(t)\right|^2, \quad\left|U^{\prime}(t)\right|^2=\left|W_{t_1}^{\prime}(t)\right|^2 .
  $$
  Furthermore, according to the assumptions of the theorem, $K\left(\gamma^{\prime}, W_{t_1}\right) \leq \tilde{K}\left(\tilde{\gamma}^{\prime}, U\right)$, so we have 
  $$
  I_0^{t_1}\left(W_{t_1}, W_{t_1}\right) \geq \tilde{I}_0^{t_1}(U, U) .
  $$
  Therefore, combining it with equations (6), we obtain:
  $$
  I_0^{t_1}\left(W_{t_1}, W_{t_1}\right) \geq \tilde{I}_0^{t_1}\left(\tilde{W}_{t_1}, \tilde{W}_{t_1}\right),
  $$
  which is equivalent to the inequality (5). Therefore, we have shown that for any $t \in (0, r)$
  $$
  \frac{\left\langle J^{\prime}(t), J(t)\right\rangle}{\langle J(t), J(t)\rangle} \geq \frac{\left\langle\tilde{J}^{\prime}(t), \tilde{J}(t)\right\rangle}{\langle\tilde{J}(t), \tilde{J}(t)\rangle}.
  $$
\end{proof}
\begin{remark}
  Rauch Comparison Theorem also holds when $J(0),\tilde{J}(0)\neq 0$ or moreover, $J$ and $\tilde{J}$ are not normal Jacobi fields. 
  In that case, $J$ and $\tilde{J}$ should satisfy: 
  $$
  \begin{aligned}
    & \left\langle J(0), \gamma^{\prime}(0)\right\rangle=\left\langle\tilde{J}(0), \tilde{\gamma}^{\prime}(0)\right\rangle, \\
    & \angle \left(J^{\prime}(0), \gamma^{\prime}(0)\right)=\angle\left(\tilde{J}^{\prime}(0), \tilde{\gamma}^{\prime}(0)\right).
  \end{aligned}  
  $$
  For more details see \cite{ChenWeiHuan2002}, Theorem 5.1, p. 342. 
\end{remark}

\bibliographystyle{unsrt}
\bibliography{NoteRef}

\end{document}